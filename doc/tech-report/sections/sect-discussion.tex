Previous sections have presented the motivation of this work, a vocabulary designed on top of the 
RDF Data Cube Vocabulary for modeling quantitative indexes and a Java-SPARQL based processor 
to validate and compute any kind of index. According to the requirements and features 
of quantitative indexes an initial evaluation is presented in Table~\ref{tab:eval-rdfindex} to show how the application of semantic technologies can ease indexes management and their computation process 
to make the most of data~\footnote{http://fcw.com/articles/2013/06/11/data-performance-management.aspx}: 
collect, verify, analyze, apply, share, protect, archive and re-use. Furthermore it is relevant to remark that data quality and filtering is 
currently a big challenge due to the vast amount of data that is continuously being generated. Policy-makers as agents in charge 
of making decisions must be able to manage this information in a timely and flexible fashion. In this context semantic technologies provides the adequate and standardized building blocks to improve the dynamism and updating capabilities of policy-maker tools. Nevertheless 
the initial effort to translate existing index definitions and computation processes to this approach can 
be hard and time-consuming but going beyond of that further updates and tools can perfectly benefit from this effort as 
other ``semantized'' domains have already done. 

 
\begin{table}[!htb]
\renewcommand{\arraystretch}{1.3}
\scriptsize
\begin{center}
\begin{tabular}{|p{3cm}|p{3cm}|p{6.5cm}|}
\hline
  \textbf{Feature} & \textbf{Crucial Step} &\textbf{Main advantages}  \\  \hline
  Data sources &  \begin{itemize} \item Collect. \item Verify. \item Tag (enrichment).\end{itemize} & \begin{itemize}
		      \item Common and shared data model, RDF.
		      \item Description of data providers (provenance and trust).
		      \item A formal query language to query data, SPARQL.
		      \item Use of Internet protocols, HTTP.
		      \item Data enrichment and validation (domain and range).
		      \item Unique identification of entities, concepts, etc. through (HTTP) URIs.
		      \item Possibility of publishing new data under the aforementioned characteristics.
		      \item Standardization and integration of data sources.
		    \end{itemize} \\ \hline  
  Structure & \begin{itemize} \item Verify. \item Tag (enrichment). \item Analyze. \item Share. \item Archive. \item Re-use. \end{itemize} & \begin{itemize}
                  \item Meta-description of index structure (validation).
                  \item Re-use of existing semantic web vocabularies.
                  \item Re-use of existing datasets to enrich meta-data.
                  \item Context-aware definitions.
                  \item Underlying logic formalism.                  
                  \item Orthogonal and flexible.                 
                 \end{itemize} \\ \hline  
  Computation process &  \begin{itemize} \item Verify. \item Tag (enrichment). \item Apply. \item Share. \item Archive. \item Re-use. \end{itemize} & \begin{itemize}
                  \item Meta-description of datasets aggregation.
                  \item Validation of composed datasets.
                  \item OWA operators support.
                  \item Direct translation to SPARQL queries.  
                 \end{itemize} \\ \hline
  Documentation & \begin{itemize} \item Verify. \item Apply. \item Share. \item Archive. \item Re-use. \end{itemize} &\begin{itemize}
                  \item Multilingual support to describe datasets, etc.
                  \item Easy generation with existing tools.                
                 \end{itemize} \\ \hline      
                 
                 Cross-Domain Features & All & \begin{itemize}
                  \item Separation of concerns and responsibilities: data and meta-data (structure and computation).
                  \item Standardization (put in action specs from organisms such as W3C).
                  \item Declarative and adaptive approach.
                  \item Non-vendor lock-in (format, access and computation).
                  \item Integration and Interoperability.
                  \item Transparency.
                  \item Help to build own indexes.
                  \item Align to existing trend (data management: quality and filtering)
                  \item Easy integration with third-party services such as visualization.
                  \item Contribution to the Web of Data.
                 \end{itemize} \\ \hline        
       
  \hline
  \end{tabular}
  \caption{Initial evaluation of semantic technologies for modeling and computing quantitative indexes.}
  \label{tab:eval-rdfindex}
  \end{center}
\end{table} 



% \begin{table}[!htb]
% \renewcommand{\arraystretch}{1.3}
% \scriptsize
% \begin{center}
% \begin{tabular}{|p{3.5cm}|p{1.8cm}|p{8.5cm}|}
% \hline
%   \textbf{Feature} & \textbf{Crucial Step} &\textbf{Main advantages}  \\  \hline
%          
%   \hline
%   \end{tabular}
%   \caption{Cross-Domain initial evaluation of semantic technologies for modeling and computing quantitative indexes.}
%   \label{tab:eval-rdfindex-cross}
%   \end{center}
% \end{table} 


 