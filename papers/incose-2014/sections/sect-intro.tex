The use of requirements in software and systems engineering lifecycles is a widely accepted practice for analysis~\cite{Sommerville:1998:REP:552009} and 
quality assessment~\cite{4685703} purposes. Traditionally requirements try to address or find out what the 
target system is supposed to do. In this sense requirements extraction as a common practice for understanding stakeholder’s 
intentions is still far from being fully developed and in most of cases the use of requirements is isolated 
in the development process. That is why the software community is trying to elevate the meaning of 
natural language specifications through new but well-known approaches such as the realization of 
Model-Driven Engineering (MDE)~\cite{DBLP:conf/gttse/Bezivin06,DBLP:conf/jisbd/Bezivin09} in combination 
with Requirements Engineering (RE), as a basic unit of specification, to support the whole lifecycle of a product or service. Furthermore and 
in case of critical systems, Software Requirements Specifications (SRS) are widely recognized as  
a key-factor for the sucess of the development process. To do so the software community has been 
involved in the research of techniques and methods for producing high-quality SRS using 
structured and controlled languages, boilerplates or patterns to guide the creation of requirements or 
ontologies to name a few. This effort is strongly justified in the field of systems engineering since 
requirements are expected to be used in the validation and verification phases of critical systems. That is 
why a proper definition of requirements must be supported to accomplish with both stakeholder's expectations and 
software quality standards such as ISO/IEC TR 24766:2009, 25000:2005, 25012:2008 or 25030:2007 (250XX) or ISO quality standards 
such as ISO/IEC 9126-1. 

In order to shift the paradigm of development processes from document-centric approaches to a 
Model-Driven Requirements Engineering (MDRE) process it is necessary to deal with the creation of requirements and, therefore, 
with the ambiguity and risks of natural language: 
\begin{itemize}
 \item Requirements are expressed with different levels of detail and precision to specify similar functionalities.
 \item Requirements may be too abstract and not easy to use by developers.
 \item Not adequate people can be involved in the creation of requirements.
 \item \ldots
\end{itemize}

To tackle this situation the use of controlled vocabularies and indicators to write \textit{better} requirements are 
common practices in the software and systems engineering areas. Furthermore a quality model is usually applied to validate the quality 
of requirements specifications to compile just in one quantitative value a set of quality indicators. Nevertheless 
existing development methodologies are facing a major challenge (interoperability) when data and information of different processes 
and tools must be shared among applications. In this context the Open Services for Lifecycle Collaboration (OSLC) 
initiative is seeking for new methods to easily integrate software and build an ideal development and operations environment.  
This initiative is releasing several specifications (leaded by industry partners) to model shared resources 
between applications with the aim of sharing more data, boosting the use of web-based standards (RDF and HTTP) 
and delivering robust products through the collaboration of development and operational tools. The OSLC as an integration 
bus is expected to be the next big thing in systems development elevating the meaning of information resources through 
semantic-based technologies. More specifically the OSLC Requirements and Quality management specifications are defining 
the appropiate concepts and properties to share data among development tools. 

On the other hand, public and private bodies are continuously seeking for new analytical tools and methods to assess some policy, change, specification, etc. and 
to support their decisions. Nevertheless the sheer mass of data now available is raising a new dynamic and challenging environment 
in which traditional tools are facing major problems to deal with data-sources diversity, structural issues or complex processes of estimation.
New paradigms and tools are required to take advantage of the existing data-based environment to design and estimate actions 
in this dynamic context according to requirements of transparency, standardization, adaptability and extensibility among others with the aim of providing new context-aware 
and added-value services that can help a deepen and broaden understanding of the impact of a policy (or a requirement change) in a more fast and efficient way. 
As a consequence common features and requirements can be extracted from the existing situation out:
\begin{itemize}
 \item Data sources. Data and information is continuously being generated as observations from social networks, public and private institutions, NGOs, services and applications, etc. 
 creating a tangled environment of sources, formats and access protocols with a huge but restricted potential for exploitation. Nevertheless data processing, knowledge inferring, etc. are not mere processes 
 of gathering and analyzing, it is necessary to deal with semantic and syntactic issues, e.g. particular measurements and dimensions or name mismatches, 
 in order to enable a proper data/information re-use and knowledge generation.
 
 \item Structure. Quantitative indexes are usually defined (a mathematical model) by experts to aggregate several indicators (in a hierarchy structure) in just one value to provide
 a measure of the impact or performance of some policy in a certain context. The structure of these indexes are obviously subjected to change over time 
 to collect more information or adjust their composition and relationships (narrower/broader). That is why technology should be able to afford 
 adequate techniques to automatically populate new changes in an efficient way.
 
  \item Computation process. This feature refers to the calculation of the index. Observations are gathered from diverse data sources and aligned 
  to the index structure, commonly indicators, that are processed through various mathematical operators to generate a final index value. 
  Nevertheless the computation process is not always described neither open (any minor change can imply a long time for validation) implying that 
  cannot be easily replied for third-parties with other purposes, for instance research, preventing one 
  of the most wanted characteristics such as transparency. Furthermore it is necessary to ensure that the computation process 
  is sound and correct.

  \item Documentation.New policy-making or quality managmeent strategies go ahead of a simple and closed value and it is necessary to bring 
  new ways of exploiting data and information. Moreover the use of the Web-based protocols as a dissemination channel represents a powerful environment in which 
  information should be available taking into account the multilingual and multicultural character of information. In this context documentation mechanisms 
  must necessarily cover all the aforementioned features to afford a detailed explanation of a quantitative index-based policy to both policymakers 
  and final users. However existing initiatives usually generates some kind of hand-made report which is not easy to keep up-to-date and deliver 
  to the long-tail of interested third-parties.
\end{itemize}

% 
Obviously semantic-based technologies, as underlying concepts of the OSLC initiative, can partially address this challenging environment for supporting 
new quality-checking strategies. A common and shared data model based on existing standardized semantic web vocabularies and datasets can be used to 
represent quantitative indexes from both, structural and computational, points of view enabling a right exploitation of meta-data and semantics. 
That is why the present paper introduces: 1) a high-level model on top of the RDF Data Cube Vocabulary~\cite{rdf-data-cube}, a shared effort to model statistical data in RDF reusing parts 
(the cube model) of the Statistical Data and Metadata Exchange Vocabulary~\cite{sdmx} (SDMX), to represent the structure and computation of quantitative indexes and 
2) a Java-SPARQL based processor to exploit the meta-information and compute the new index values.
% 

% 
Finally as a motivating and on-going example, see Table~\ref{tab:example-wb}, a quality manager wants to re-use requirements metrics to model and compute a new index, 
``The Naive Requirement Quality Index''. This  index uses the components ``Maintainability'' ($c_1$) and ``Usability'' ($c_2$) with the aim of 
comparing the status and evolution of requirements in a project to decide whether new actions are performed. 
From these components two indicators have been respectively selected by experts: ``Maturity'' ($in_1$) and ``Understandability'' ($in_2$). 
Once components and their indicators are defined and data can be retrieved from the requirements, a quality tool it is necessary to set how the index and 
components are computed taking into account that only data about indicators is available. Following a top-down approach the index, $i$, is calculated through 
an ordered weighted averaging (OWA) operator using the formula: $\sum_{i=1}^n  w_i c_i$, where $w_i$ is the weight of the component $c_i$. On the other hand, 
both components only aggregates one indicator, the computation must also consider the observation status and values must be normalized using the 
\textit{z-score} before computing intermediate and final values for each indicator, component and index. Furthermore this index is supposed to 
change in the future adding new components or indicators, modifying the computation processes (weights) or the structure (new components, indicators and dimensions). 
Finally, the quality manager is also interested in applying this index to other scenarios and he also needs a way of explaining in different 
languages how the index is computed. 


\begin{table}[!htb]
\renewcommand{\arraystretch}{1.3}
\scriptsize
\begin{center}
\begin{tabular}{|p{2.5cm}|p{3cm}|p{2cm}|p{1cm}|p{1.5cm}|}
\hline
  \textbf{Description} & \textbf{Requirement} & \textbf{Project} & \textbf{Value} & \textbf{Status} \\  \hline
  Stability & $R_1$ & $P_1$ & $79$ & Normal \\ \hline
  Understandability & $R_1$ & $P_1$  & $85$ & Normal\\ \hline
  Stability & $R_2$ & $P_1$ & $79$ & Normal \\ \hline
  Understandability & $R_2$ & $P_1$  & $85$ & Normal\\ \hline
  \hline
  \end{tabular}
  \caption{Example of indicator observations from an ISO-based Quality Model.}
  \label{tab:example-wb}
  \end{center}	 
\end{table} 

% 
% 
% 
% 	 	
% 
% 
