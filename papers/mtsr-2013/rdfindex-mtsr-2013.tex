\documentclass{llncs}

%\usepackage{llncsdoc}

%\usepackage{makeidx}  % allows for indexgeneration
\usepackage{graphicx}
\usepackage[T1]{fontenc}
\usepackage[english]{babel}
\usepackage[utf8]{inputenc}
\usepackage{multirow}

\usepackage{url}
\usepackage{rotating}

%%%Math
\usepackage{latexsym}
% \usepackage{amsmath}
% \usepackage{amssymb}
% \usepackage{amsthm}
%\usepackage{eurosans}

\usepackage{eurosym}

\usepackage{longtable}

\usepackage{listings}

\usepackage{color}
\usepackage{textcomp}


\definecolor{gray}{gray}{0.5}
\definecolor{green}{rgb}{0,0.5,0}


\begin{document}
\title{Leveraging semantics to represent and compute quantitative indexes. \\ The RDFindex approach.}

\author{Jose Mar\'{i}a \'{A}lvarez\inst{1} \and Jos\'{e} Emilio Labra\inst{2}  \and Patricia Ordoñez de Pablo\inst{3}}


\authorrunning{Jose Mar\'{i}a Alvarez \and  Jos\'{e} Emilio Labra \and Patricia Ordoñez de Pablos}


\institute{South East European Research Center, \\ 54622, Thessaloniki, Greece\\
  \email{{jmalvarez@seerc.org}}\\
\and WESO Research Group, Department of Computer Science, University of Oviedo, \\ 33007, Oviedo, Spain.\\
  \email{{labra@uniovi.es.es}}\\  
  \and WESO Research Group, Department of Business Administration, University of Oviedo, \\ 33007, Oviedo, Spain.\\
  \email{{patriop@uniovi.es.es}}\\  
}


\date{}

\maketitle

\renewcommand{\labelitemi}{$\bullet$}

\begin{abstract}
FIXME
\end{abstract}

\section{Introduction}
Public and private bodies are continuously seeking for new analytical tools and methods to assess, rank and compare their performance based 
on different indicators and dimensions with the objective of making some decision or developing a new policy. 
In this context the creation and use of quantitative indexes is a wider but critized practice that has been applied to various 
domains such as bibliometrics and academic performance and quality (the Impact Factor by Thomson-Reuters or the H-index), 
the Web impact (the Webindex by the Webfoundation) or Cloud Computing (the CSMIC index, the Global Cloud Index by Cisco, the CSC index, 
the VMWare Cloud Index, etc.) or university quality (the Shanghai or the Webometrics rankings) to name a few. 
Therefore policy-makers as well as individuals are continuously evaluating quantitative measures to tackle 
existing problems or support their decisions. Nevertheless the sheer mass of data now available in the web is 
raising a new dynamic and challenging environment in which traditional tools are facing major 
problems to deal with datasources diversity, structural issues or complex processes of estimation. Following some efforts 
such as ``Policy-making $2.0$'' that \textit{refers to a blend of emerging and fast developing technologies 
that enable better, more timely and more participated decision-making}, new paradigms and tools are required to take advantage of 
the existing environment (open data and big data) to design and estimate this dynamic environment according to new requirements of 
transparency, standardization, adaptability and extensibility among others with the aim of providing new context-aware 
and added-value services such as visualization that can help a deepen and broaden understanding of the impact of a 
policy in a more fast and efficient way. As a consequence common features and requirements can be extracted out from the existing situation:
\begin{itemize}
 \item Data sources. 
 \item Structure. Quantitative indexes are usually defined by experts to aggregate several indicators in just one value providing a measure of the impact or 
 performance of some policy or activity in a certain area. The structure of these indexes are obviously subjected to change over time 
 to collect more information or adjust some parameters and technology should be able to afford proper techniques 
 to automatically represent new changes in an efficient way.
 
 \item Computation process. 
 \item Documentation.
\end{itemize}


 %. That is why a technique to automatically define index structure through a common and shared model can help...



As a motivating example, see Table~\ref{tab:example-wb}, a policy-maker wants to re-use the WorldBank data to model and compute a new index, ``The Naive World Bank Index''. This 
index uses the topics ``Aid Efectiveness'' ($c_1$) and ``Health'' ($c_2$) with the aim of comparing the status and evolution of health in several countries to decide whether new 
investments are performed. From these components two indicators has been respectively selected by experts: ``Life Expectancy'' ($i_1$) and ``Health expenditure, total (\%) of GDP'' ($i_2$). 
Once components and their indicators are defined and data can be retrieved from the WorldBank it is necessary to set how the index and components are computed 
taking into account that only data about indicators is available. Following a top-down approach the index, $i$, is calculated through ordered weighted averaging (OWA) using the 
formula: $\sum_{i=1}^n  w_i c_i$, where $w_i$ is the weight of the component $c_i$. On the other hand, both components only aggregates one indicator but the ``Aid Efectiveness'' 
must firstly compute the ``Life Expectancy'' without considering the sex dimension. Making this implies the need of calculating the average age by country and year to create 
a new ``derivated'' indicator of ``Life Expectancy''. Apart from that the computation must also consider the observation status and values must be normalized using the 
z-score before computing intermediate and final values for each indicator, component and index. Furthermore this index is supposed to change in the future 
adding new data sources, modifying the computation processes (weights) or the structure (new components, indicators and dimensions). Finally, the policy-maker 
is also interested in applying this index to other scenarios and he needs a way of explaining in different languages how the index is computed. 



\begin{table}[!htb]
\renewcommand{\arraystretch}{1.3}
\begin{center}
\begin{tabular}{|p{3cm}||p{5cm}|p{1.8cm}|p{1.8cm}|p{1cm}|}
\hline
  \textbf{Component} & \textbf{Indicator} & \textbf{Year} & \textbf{Country} & \textbf{Value}  \\  \hline
  Aid Efectiveness & Life Expectancy Male & 2010 & Spain & $1.0$ \\ \hline
  Aid Efectiveness & Life Expectancy Male & 2011 & Spain & $1.0$ \\ \hline
  Aid Efectiveness & Life Expectancy Female & 2010 & Spain & $1.0$ \\ \hline
  Aid Efectiveness & Life Expectancy Female & 2011 & Spain & $1.0$ \\ \hline
  Aid Efectiveness & Life Expectancy Male & 2010 & Greece & $1.0$ \\ \hline
  Aid Efectiveness & Life Expectancy Male & 2011 & Greece & $1.0$ \\ \hline
  Aid Efectiveness & Life Expectancy Female & 2010 & Greece & $1.0$ \\ \hline
  Aid Efectiveness & Life Expectancy Female & 2011 & Greece & $1.0$ \\ \hline
  Health & Health expenditure, total (\%) of GDP) & 2010 & Spain & $1.0$ \\ \hline
  Health & Health expenditure, total (\%) of GDP) & 2011 & Spain & $1.0$ \\ \hline
  Health & Health expenditure, total (\%) of GDP) & 2010 & Spain & $1.0$ \\ \hline
  Health & Health expenditure, total (\%) of GDP) & 2011 & Spain & $1.0$ \\ \hline
  \hline
  \end{tabular}
  \caption{Example of observations from the WorldBank.}
  \label{tab:example-wb}
  \end{center}
\end{table} 


\section{Related Work}

%In the particular case of statistical data, the RDF Data Cube Vocabulary (Cyganiak & Reynolds, 2012), a W3C Working Draft document, is a shared effort to represent statistical data in RDF reusing parts (the cube model) of the Statistical Data and Metadata Exchange Vocabulary (SDMX) (SDMX initiative, 2005), an ISO standard for exchanging and sharing statistical data and metadata among organizations. The Data Cube vocabulary is a core foundation which supports extension vocabularies to enable publication of other aspects of statistical data flows or other multi-dimensional data sets. Previously, the Statistical Core Vocabulary (SCOVO, 2009), published by DERI, was the standard in fact to describe statistical information in the Web of Data. Some works are also emerging to publish statistical data following the concepts of the LOD initiative such as (Bosch, Cyganiak, Wackerow et al., 2012), (Zapilko & Mathiak, 2012) or (Rivera-Salas, Martin,  Maia Da Mota,  Auer, S. et al., 2012) among others.


\section{Theoretical modelling of a quantitative composite index}
\section{Representation of a quantitative composite index in RDF: The RDFIndex}
\section{A $\{Java\rightarrow based \rightarrow SPARQL\}$  interpreter of the RDFIndex}
%
\subsection{Use Case: FIXME}
% % 
\section{Discussion and Further Steps}

\section{Conclusions and Future Work}

\end{document}

