\documentclass{llncs}

%\usepackage{llncsdoc}

%\usepackage{makeidx}  % allows for indexgeneration
\usepackage{graphicx}
\usepackage[T1]{fontenc}
\usepackage[english]{babel}
\usepackage[utf8]{inputenc}
\usepackage{multirow}

\usepackage{url}
\usepackage{rotating}

%%%Math
\usepackage{latexsym}
%\usepackage{amsmath}
%\usepackage{amssymb}
%\usepackage{amsthm}
%\usepackage{eurosans}

\usepackage{eurosym}

\usepackage{longtable}

\usepackage{listings}

\usepackage{color}
\usepackage{textcomp}



\definecolor{gray}{gray}{0.5}
\definecolor{green}{rgb}{0,0.5,0}


\begin{document}

\title{Leveraging semantics to represent and compute quantitative indexes. \\ The RDFIndex approach.\thanks{This is the full version of the article including some source 
code listings, tables or complete examples for a better understanding during the reviews. Nevertheless some listings, tables, examples and figures will be omitted in a final version.}}

\author{Jose Mar\'{i}a \'{A}lvarez\inst{1} \and Jos\'{e} Emilio Labra\inst{2}  \and Patricia Ordoñez de Pablos\inst{3}}


\authorrunning{Jose Mar\'{i}a Alvarez \and  Jos\'{e} Emilio Labra \and Patricia Ordoñez de Pablos}


\institute{South East European Research Center, \\ 54622, Thessaloniki, Greece\\
  \email{{jmalvarez@seerc.org}}\\
\and WESO Research Group, Department of Computer Science, University of Oviedo, \\ 33007, Oviedo, Spain.\\
  \email{{labra@uniovi.es.es}}\\  
  \and WESO Research Group, Department of Business Administration, University of Oviedo, \\ 33007, Oviedo, Spain.\\
  \email{{patriop@uniovi.es.es}}\\  
}


\date{}

\maketitle

\renewcommand{\labelitemi}{$\bullet$}

\begin{abstract}
The compilation of key performance indicators (KPIs) in just one value is 
becoming a challenging task in certain domains to summarize data and information. 
In this context, policymakers are continuously gathering and analyzing statistical 
data with the aim of providing objective measures about a specific policy, activity, 
product or service and making some kind of decision. Nevertheless existing tools 
and techniques based on traditional processes are preventing a 
proper use of the new dynamic and data environment avoiding more timely, 
adaptable and flexible (on-demand) quantitative index creation. On the other hand, 
semantic-based technologies emerge to provide the adequate building blocks 
to represent domain-knowledge and process data in a flexible fashion 
using a common and shared data model. That is why a RDF vocabulary designed on 
the top of the RDF Data Cube Vocabulary to model quantitative indexes 
is introduced in this paper. Moreover a Java and SPARQL based processor 
of this vocabulary is also presented as a tool to exploit the index meta-data structure and automatically 
perform the computation process to populate new values. Finally some discussion, 
conclusions and future work are also outlined.
\end{abstract}
% 
\section{Introduction}
Public and private bodies are continuously seeking for new analytical tools and methods to assess, rank and compare their performance based 
on different indicators and dimensions with the objective of making some decision or developing a new policy. 
In this context the creation and use of quantitative indexes is a wider but critized practice that has been applied to various 
domains such as bibliometrics and academic performance and quality (the Impact Factor by Thomson-Reuters or the H-index), 
the Web impact (the Webindex by the Webfoundation) or Cloud Computing (the CSMIC index, the Global Cloud Index by Cisco, the CSC index, 
the VMWare Cloud Index, etc.) or university quality (the Shanghai or the Webometrics rankings) to name a few. 
Therefore policy-makers as well as individuals are continuously evaluating quantitative measures to tackle 
existing problems or support their decisions. Nevertheless the sheer mass of data now available in the web is 
raising a new dynamic and challenging environment in which traditional tools are facing major 
problems to deal with datasources diversity, structural issues or complex processes of estimation. Following some efforts 
such as the ``Policy-making $2.0$'' within the Cross-Over project that \textit{refers to a blend of emerging and fast developing technologies 
that enable better, more timely and more participated decision-making}, new paradigms and tools are required to take advantage of 
the existing environment (open data and big data) to design and estimate this dynamic environment according to new requirements of 
transparency, standardization, adaptability and extensibility among others with the aim of providing new context-aware 
and added-value services such as visualization that can help a deepen and broaden understanding of the impact of a 
policy in a more fast and efficient way. As a consequence common features and requirements can be extracted out from the existing situation:
\begin{itemize}
 \item Data sources. Data and information is continuously being generated as observations from social networks, public and private institutions, NGOs, services and applications, etc. 
 creating a tangled environment of sources, formats and access protocols with a huge potential for exploitation. Nevertheless data processing, knowledge inferring, etc. are not mere processes 
 of retrieving and gathering, it is necessary to deal with semantic and syntactic issues with the aim of enabling a proper re-use of data and information 
 and generate new knowledge. 
 
 \item Structure. Quantitative indexes are usually defined (a mathematical model) by experts to aggregate several indicators (in a tree structure) in just one value providing 
 a measure of the impact or performance of some policy in a certain context. The structure of these indexes are obviously subjected to change over time 
 to collect more information or adjust some parameters. That is why technology should be able to afford the proper techniques 
 to automatically represent new changes in an efficient way.
 
  \item Computation process. This feature refers to the calculation of the index. Observations are gathered from different data sources and aligned 
  to the index structure, commonly indicators, that are processed through different mathematical operators to compute the final value of the index. 
  Nevertheless the computation process is not always open and can not be replied for third-parties with other purposes, for instance research, preventing some 
  of most wanted characteristics such as transparency. Furhermore it is not possible to assess if the index is going to be or has been computed in 
  the right way.
  
  \item Documentation. As the European project Cross-over has stated, new policy-making strategies go ahead of a simple and closed value and it is necessary to provide 
  new ways of exploiting data and information. Moreover the use of the Web as a dissemination channel represents a powerful environment in which 
  information should be available taking into account multilingual and multicultural character of information. In this context documentation mechanisms 
  must necessarily cover all the aforementioned features to afford a detailed explaination of a quantitative index-based policy to both policy-makers 
  and final users but...
\end{itemize}

On the other hand, the Semantic Web area has experienced during last years a growing commitment from both academia and industrial areas 
with the objective of elevating the meaning of web information resources through a common and shared data model (graphs) and 
an underlying semantics based on different logic formalisms (ontologies). The Resource Description Framework (RDF), based on a graph model, 
and the Web Ontology Language (OWL), designed to formalize and model domain knowledge, are a \textit{lingua-franca} to reuse information 
and data in a knowledge-based environment. Thus data, information and knowledge can be easily shared, exchanged and linked~\cite{Maali_Cyganiak_2011} 
to other knowledge and databases through the use URIs, more specifically HTTP-URIs. Therefore the broad objective of this effort can 
be summarized as a new environment of added-value services that can encourage and improve B2B (Business to Business), B2C (Business to Client) or 
A2A (Administration to Administration) relationships by means of the implementation of new contex-awareness expert systems 
to tackle existing cross-domain problems in which data heterogeneities, lack of standard knowledge representation and 
interoperability problems are common factors. As part of the Semantic Web area, recent times have also seen the deployment of 
the Linked (Open) Data initiative  to make it possible the view and application of the Semantic Web to create a large 
and distributed database on the Web. 

Obviously semantic web technologies and tools can partially fulfill the features and requirements of this challenging environment for supporting 
new policy-making strategies. A common and shared data model based on existing standardized semantic web vocabularies and datasets can be used to 
represent quantitative indexes from both, structural and computational, points of view enabling the right exploitation of metadata and semantics. 
That is why the present paper introduces: 1) a high-level model on top of the RDF Data Cube Vocabulary, a shared effort to model statistical data in RDF reusing parts 
(the cube model) of the Statistical Data and Metadata Exchange Vocabulary (SDMX), to represent the structure of quantitative indexes and 
2) a Java-SPARQL based processor to exploit the metainformation and compute new values of an index.

Finally, as a motivating example, see Table~\ref{tab:example-wb}, a policy-maker wants to re-use the WorldBank data to model and compute a new index, ``The Naive World Bank Index''. This 
index uses the topics ``Aid Efectiveness'' ($c_1$) and ``Health'' ($c_2$) with the aim of comparing the status and evolution of health in several countries to decide whether new 
investments are performed. From these components two indicators has been respectively selected by experts: ``Life Expectancy'' ($in_1$) and ``Health expenditure, total (\%) of GDP'' ($in_2$). 
Once components and their indicators are defined and data can be retrieved from the WorldBank it is necessary to set how the index and components are computed 
taking into account that only data about indicators is available. Following a top-down approach the index, $i$, is calculated through ordered weighted averaging (OWA) using the 
formula: $\sum_{i=1}^n  w_i c_i$, where $w_i$ is the weight of the component $c_i$. On the other hand, both components only aggregates one indicator but the ``Aid Efectiveness'' 
must firstly compute the ``Life Expectancy'' without considering the sex dimension. Making this implies the need of calculating the average age by country and year to create 
a new ``derivated'' indicator of ``Life Expectancy''. Apart from that the computation must also consider the observation status and values must be normalized using the 
z-score before computing intermediate and final values for each indicator, component and index. Furthermore this index is supposed to change in the future 
adding new data sources, modifying the computation processes (weights) or the structure (new components, indicators and dimensions). Finally, the policy-maker 
is also interested in applying this index to other scenarios and he needs a way of explaining in different languages how the index is computed. 


\begin{table}[!htb]
\renewcommand{\arraystretch}{1.3}
\begin{center}
\begin{tabular}{|p{3cm}||p{4cm}|p{1.8cm}|p{1.8cm}|p{1cm}|}
\hline
  \textbf{Component} & \textbf{Indicator} & \textbf{Year} & \textbf{Country} & \textbf{Value}  \\  \hline
  Aid Efectiveness & Life Expectancy Male & 2010 & Spain & $1.0$ \\ \hline
  Aid Efectiveness & Life Expectancy Male & 2011 & Spain & $1.0$ \\ \hline
  Aid Efectiveness & Life Expectancy Female & 2010 & Spain & $1.0$ \\ \hline
  Aid Efectiveness & Life Expectancy Female & 2011 & Spain & $1.0$ \\ \hline
  Aid Efectiveness & Life Expectancy Male & 2010 & Greece & $1.0$ \\ \hline
  Aid Efectiveness & Life Expectancy Male & 2011 & Greece & $1.0$ \\ \hline
  Aid Efectiveness & Life Expectancy Female & 2010 & Greece & $1.0$ \\ \hline
  Aid Efectiveness & Life Expectancy Female & 2011 & Greece & $1.0$ \\ \hline
  Health & Health expenditure, total (\% of GDP) & 2010 & Spain & $1.0$ \\ \hline
  Health & Health expenditure, total (\% of GDP) & 2011 & Spain & $1.0$ \\ \hline
  Health & Health expenditure, total (\% of GDP) & 2010 & Spain & $1.0$ \\ \hline
  Health & Health expenditure, total (\% of GDP) & 2011 & Spain & $1.0$ \\ \hline
  \hline
  \end{tabular}
  \caption{Example of observations from the WorldBank.}
  \label{tab:example-wb}
  \end{center}
\end{table} 


\section{Related Work}
%Open Data
Recent times have seen the deployment of the Open Data initiative due to the strategy 
developed by the governments in USA, Australia, UK or Europe and that have been followed by most of countries around the world to deliver data 
catalogues containing valuable public sector information (PSI). In this context the Open (Government) Data movement (W3C) is making a great effort 
to spread this view in public and private bodies with the objective of boosting transparency and a new data-driven economy. 
From a corporate strategy point of view the re-use of existing open data must encourage and improve 
more efficient policy-making processes. Under this view there is a perfect matching between the Linked Data and the Open Data 
principles: on the one hand the Linked Data community is generating the proper environment of standard 
specifications and tools to manage data and, on the other hand, the Open Data initiative is requiring 
the right methods to generate an authentic re-use of public data. The combination of these 
two views leads us to the Linked Open Data (LOD) effort that seeks for applying the principles of Linked Data to implement the Open Data mission.

Currently one of the mainstreams in the Semantic Web area lies in the application of the LOD initiative in 
different domains such as  e-Government, e-Procurement, e-Health, Biomedicine, Education, Bibliography or Geography to name a few,  
with the aim of solving existing problems of integration and interoperability among applications and create a 
knowledge environment under the Web-based protocols. In this context, the present work is therefore focused 
in applying semantic web vocabularies and datasets to model quantitative indexes from both structural 
and computational points of view in a ``Policy-Making'' context. In order to reach the major objective of building a re-usable Web of Data,  
the publication of information and data under a common data model (RDF) and formats with 
a specific formal query language (SPARQL~\cite{Sparql11}) provide the required building blocks to turn 
the Web of documents into a real database of data. As a consequence the popular 
diagram of the LOD Cloud, generated from meta-data extracted from the Comprehensive Knowledge Archive Network (CKAN) out, 
contains $337$ datasets, with more than $25$ billion RDF triples and $395$ million links in different  domains. 
Research works are focused in two main areas: 1) production/publishing~\cite{bizer07how} and 2) consumption. 
In the first case data quality~\cite{wiqa,lodq}, conformance~\cite{HoganUHCPD:2012:237}, 
provenance~\cite{w3c-prov}, trust~\cite{Carroll05namedgraphs}, description of datasets~\cite{void} and 
entity reconciliation~\cite{Maali_Cyganiak_2011} are becoming major objectives since a mass of amount data is already available 
through SPARQL endpoints deployed on the top of RDF repositories such as OpenLink Virtuoso or OWLim. 

On the other hand, consumption of Linked Data is being addressed to provide new ways of data visualization~\cite{DBLP:journals/semweb/DadzieR11}, 
faceted browsing~\cite{Pietriga06fresnel} and searching~\cite{hoga-etal-2011-swse-JWS}, processing~\cite{Harth:2011:SIP:1963192.1963318} and exploitation of data applying 
different approaches such as sensors~\cite{Jeung:2010:EMM:1850003.1850235} and techniques  such as distributed 
queries\cite{Hartig09executingsparql}, scalable reasoning process~\cite{DBLP:conf/semweb/HoganPPD10}, 
annotation of web pages~\cite{rdfa-primer} or information retrieval~\cite{Pound} to name a few.

In the particular case of statistical data, the RDF Data Cube Vocabulary~\cite{rdf-data-cube}
a W3C Working Draft document, is a shared effort to represent statistical data in RDF reusing parts (the cube model) 
of the Statistical Data and Metadata Exchange Vocabulary (SDMX)~\cite{sdmx}, an ISO standard 
for exchanging and sharing statistical data and meta-data among organizations. The Data Cube vocabulary is a core 
foundation which supports extension vocabularies to enable publication of other aspects of statistical data flows or 
other multi-dimensional data sets. Previously, the Statistical Core Vocabulary~\cite{scovo} was the standard in fact to describe statistical information in the Web of Data.
Some works are also emerging to mainly publish statistical data following the concepts of the LOD initiative 
such as~\cite{DBLP:conf/semweb/ZapilkoM11,DBLP:journals/ijsc/SalasMBCMA12,DDI2013,DBLP:conf/dgo/FernandezMG11,webindexlod} 
among others.

All the aforementioned works must be considered in order to re-use existing vocabularies and datasets to address 
the challenges of creating meta-described quantitative indexes. Mainly semantics allows us to model logical restrictions 
on data and the computation process while linked data enables the description of indexes in terms of existing concepts and 
the publication of new data and information under a set of principles to boost their re-use and automatic 
processing through machine-readable formats and access protocols.


\section{Theoretical modeling of a quantitative composite index}
% This section outlines a model for representing quantitative indexes based on the aggregation 
% of different components and indicators. Furthermore a computation process for those elements is 
% also presented in order to specify the population of new observations. 
Essentially a quantitative index is comprised of the aggregation of several components. In the same way, 
a component is also composed of the aggregation of indicators that keep concrete observations. From this initial definition 
some characteristics and assumptions can be found: 1) although observations can be directly mapped to an index or a component, they 
are frequently computed applying a bottom-up approach from an indicator to a component and so on. 2) An observation is 
a real numerical value extracted from some agent out under a precise context. Generally observations only takes one measure and are considered 
to be raw without any pre-processing. 3) Before aggregating observation values, components and indexes can 
estimate missing values to finally normalize them in order to get a quantitative value. According to the aforementioned characteristics and assumptions an ``observable'' element (index, component or indicator) is a 
dataset of observations under a specific context (dimensions and/or meta-data) that can be directly gathered from external 
sources or computed by some kind of OWA operator. 

\begin{definition}[Observation-$o$]\upshape
It is a tuple $\{v,m,s\}$, where $v$ is a numerical value for the measure $m$ with an status $s$ that belongs to 
only one dataset of observations $O$. 
\end{definition}


\begin{definition}[Dataset-$q$]\upshape
It is a tuple $\{O,m,D,A,T\}$ where $O$ is a set of observations for only one measure $m$ that is described under 
a set of dimensions $D$ and a set of annotations $A$. Additionally, some attributes can be defined in the set $T$ for structure enrichment. 
\end{definition}


\begin{definition}[Aggregated dataset-$aq$]\upshape
It is an aggregation of $n$ datasets $q_i$ (identified by the set $Q$) which set of observations $O$ is derivate by applying 
an OWA operator $p$ to the observations $O_{q_i}$. 
\end{definition}

As a necessary condition for the computation process, an aggregated dataset $aq$ defined by means of the set of dimensions $D_{aq}$ can be computed iif 
$\forall q_j \in Q: D_{aq} \subseteq D_{q_j}$. Furthermore the OWA operator $p$ can only aggregate values belonging to the same measure $m$.

As a consequence of the aforementioned definitions some remarks must be outlined in order to restrict the understanding of 
a quantitative index (structure and computation):
\begin{itemize}
 \item The set of dimensions $D$, annotations $A$ and attributes $T$ for a given dataset $Q$ is always the same with the aim of describing all observations under 
 the same context.
 \item An index $i$ and a component $c$ are aggregated datasets. Nevertheless this restriction is relaxed if observations can be directly mapped to 
 these elements without any computation process.
 \item An indicator $in$ can be both dataset or aggregated dataset.
 \item All elements in definitions must be uniquely identified. 
 \item An aggregated dataset is also a dataset.
\end{itemize}

Following the on-going example, see Table~\ref{tab:example-wb}, the modeling of the ``The Naive World Bank Index'' would be the next one:
\begin{itemize}
 \item Each row of the table is an observation $o_i$ with a numerical value $v$, the measure is $m_{in}$ and the status is ``Raw''.
 \item Two indicators can be found: \{ ($in_1$, ``Life Expectancy''), ($in_2$, ``Health expenditure, total (\% of GDP)'') \}, each indicator contains a set 
 of observations $O_{in_i}$. The dimensions for each indicator are: $D_{in_1}$  \{(``Year'', ``Country'', ``Sex''\} and $D_{in_2}$ \{``Year'', ``Country''\}.
 \item In order to group the ``Life Expectancy'' without the ``Sex'' dimension it is necessary to define a new aggregated dataset $aq_1$ which 
 dimensions $D_{aq_1}$ are \{``Year'', ``Country''\} and the OWA operator is the average of values $v \in O_{in_1}$. In this sample the aggregated indicator $aq_1$
 can be assembled due to the indicator ``Life Expectancy'' accomplishes with the aforementioned necessary conditions: 1) $D_{aq} \subseteq D_{in_1} \wedge D_{aq_1} \subseteq D_{in_2}$ and 
 2) $m_{aq_1}= m_{in_1}$ = ``Life Expectancy''.
 \item In the same way, the set of components: \{($c_1$,``Aid Efectiveness''), ($c_2$,``Health'')\} are built aggregating the indicators $aq_1$ and 
 $q_2$ using as OWA operator the ``min'' value. In this case ``min'' or ``max'' operators can be used due to an observation is uniquely identified in a 
 dataset by a tuple $\{v,m,s\} \cup D$.
 \item Finally, the index is computed using the general form of an OWA operator $\sum_{i=1}^n  w_i c_i$ and taking as weights those we select.
\end{itemize}

As final remark, the computation process is generating new observations, following a bottom-up approach, according to the structure defined 
in each dataset. Although a logical structure of indexes, components and indicators can be directly established using narrower/broader properties 
the main advantage of this approach lies in the possibility of expressing new elements by aggregating the existing ones. Nevertheless restrictions 
about the type of dataset that can be aggregated in each level could be added at any time for other reasons such as validation or to generate 
a human-readable form of the index.


\section{Representation of a quantitative composite index in RDF: The RDFIndex}
Since previous section has stated the key definitions to represent quantitative indexes by aggregation, a direct translation built 
on the top of the RDF Data Cube Vocabulary, SDMX and other semantic web vocabularies is presented in Table~\ref{index-to-rdf}. Thus 
all concepts in the index are described reusing existing definitions, taking advantage of previous efforts and pre-established semantics 
with the aim of being extended in the future to fit new requirements. According to these mappings a definition of the index in the 
on-going example and some dimensions are presented in Figure~\ref{fig:results-rdf-index} and Figure~\ref{fig:results-rdf-properties}.

% FIXME: Maybe an observation? include a table with all reused vocabularies? a diagram of the ontology?

\begin{table}[!htb]
\renewcommand{\arraystretch}{1.3}
\scriptsize
\begin{center}
\begin{tabular}{|p{3cm}||p{4.8cm}|p{4.5cm}|}
\hline
  \textbf{Concept} & \textbf{Vocabulary element} &  \textbf{Comments}  \\  \hline
   Observation $o$ & \texttt{qb:Observation} &  Enrichment through annotations\\ \hline
   Numerical value $v$ & \texttt{xsd:double} & Restriction to numerical values  \\ \hline
   Measure $m$ & \texttt{qb:MeasureProperty} \texttt{sdmx-measure:obsValue} & Restriction to one measure \\ \hline
   Status $s$ & \texttt{sdmx-concept:obsStatus} & Defined by the SDMX-RDF vocabulary\\ \hline
   Dataset $q$ & \texttt{qb:dataSet} and \texttt{qb:qb:DataStructureDefinition} &  Metadata of the \texttt{qb:dataSet}\\ \hline
   Dimension $d_i \in D$ & \texttt{qb:DimensionProperty} & Context of observations \\ \hline
   Annotation $a_i \in A$ & \texttt{owl:AnnotationProperty} &  Intensive use of Dublin Core\\ \hline
   Attribute $at_i \in T$ & \texttt{qb:AttributeProperty} & Link to existing datasets such as DBPedia \\ \hline
   OWA operator $p$ &  SPARQL 1.1 aggregation operators & Other extensions depend on the RDF repository \\ \hline
   Index, Component and Indicator & \texttt{skos:Concept} & SKOS taxonomy (logical structure) \\ \hline
  \hline
  \end{tabular}
  \caption{Summary of mappings between the index definition and the RDF Data Cube Vocabulary.}
  \label{index-to-rdf}
  \end{center}
\end{table} 

\begin{figure}[!ht]
\begin{lstlisting}[language=XML,basicstyle=\scriptsize]  
@prefix rdfindex:  <http://purl.org/rdfindex/ontology/> .
@prefix rdfindex-wb:  <http://purl.org/rdfindex/wb/resource/> .
@prefix rdfindex-wbont:  <http://purl.org/rdfindex/wb/ontology/> .

rdfindex-wb:TheWorldBankNaiveIndex 
  a rdfindex:Index;
  rdfs:label "The World Bank Naive Index"@en;
  rdfindex:type rdfindex:Quantitative;
  rdfindex:aggregates [ 		
    rdfindex:aggregation-operator rdfindex:OWA;
    rdfindex:part-of [
      rdfindex:dataset rdfindex-wb:AidEffectiveness; 
      rdfindex:weight 0.4];
    rdfindex:part-of [rdfindex:dataset rdfindex-wb:Health; 
      rdfindex:weight 0.6];
  ];
  #More meta-data properties...
  qb:structure 	rdfindex-wb:TheWorldBankNaiveIndexDSD ; .
  
rdfindex-wb:TheWorldBankNaiveIndexDSD 
  a qb:DataStructureDefinition;  
  qb:component    
  [qb:dimension rdfindex-wbont:ref-area],
  [qb:dimension rdfindex-wbont:ref-year],
  [qb:measure   rdfindex:value],
  [qb:attribute sdmx-attribute:unitMeasure];
  #More meta-data properties...
  .
\end{lstlisting}
\caption{Example of an index structure in RDF.}
 \label{fig:results-rdf-index}
\end{figure}

\begin{figure}[!ht]
\begin{lstlisting}[language=XML,basicstyle=\scriptsize]  
rdfindex-wbont:ref-area a rdf:Property, 
  qb:DimensionProperty; 
   rdfs:subPropertyOf sdmx-dimension:ref-area; 
  rdfs:range skos:Concept; 
  qb:concept sdmx-concept:ref-area . 

rdfindex:value a rdf:Property, qb:MeasureProperty;
  rdfs:label "Value of an observation"@en;
  skos:notation "value" ;
  rdfs:subPropertyOf sdmx-measure:obsValue;
  rdfs:range xsd:double . 
\end{lstlisting}
\caption{Example of a dimension and a measure definition in RDF.}
 \label{fig:results-rdf-properties}
\end{figure}


Since the structure and the computation processes can be built on the top of existing RDF vocabulary it is then 
possible to make a translation to a generic SPARQL query (includes the basic OWA operator), see Figure~\ref{fig:results-rdf-sparql-template}, 
in order generate new observations following the bottom-up approach that previous section has outlined.

\begin{figure}[!ht]
\begin{lstlisting}[language=SQL,basicstyle=\scriptsize,mathescape]  
SELECT ($d_i \in D$) [(sum(?w*?measure) as ?newvalue) | OWA(?measure)]
WHERE{
  $q$ rdfindex:aggregates ?parts.
  ?parts rdfindex:part-of ?partof.
  ?partof rdfindex:dataset $q_i$ .
  FILTER($?partof \in Q$).  
  ?observation rdf:type qb:Observation.
  ?part rdfindex:weight ?defaultw.     
  OPTIONAL {?partof rdfindex:weight ?aggregationw.}.
  BIND (if( BOUND(?aggregationw), ?aggregationw, ?defaultw) AS ?w)
  ?observation $m$ ?measure . 
  ?observation ?dim ?dimRef. 
  FILTER ($?dim  \in D$).
}GROUP BY ($d_i \in D$)
\end{lstlisting}
\caption{SPARQL template for building aggregated observations.}
 \label{fig:results-rdf-sparql-template}
\end{figure}


\subsection{A Java-SPARQL based interpreter of the RDFIndex}
The first implementation of the RDFIndex vocabulary processor~\footnote{\url{https://github.com/chemaar/rdfindex}} is based on traditional 
language processor techniques such as the use of design patterns (e.g. Composite or Visitor) 
to separate the exploitation of meta-data from the interpretation. Thus the processor 
works and provides next functionalities (hereafter load and query an endpoint are completely equivalent due to 
data is separated from access and storage formats):

\begin{itemize}
 \item The RDFIndex ontology is loaded to have access to common definitions.
 \item The structure of an index described with the aforementioned vocabulary is 
 also loaded to create a kind of Abstract Syntax Tree (AST) containing the defined meta-data.
 \item Once the meta-data structure is available in the AST it can be managed 
 through three AST walkers that performs: 1) validation (structure and RDF Data Cube normalization) 
 and 2) SPARQL queries creation and 3) documentation generation (optional). 
%  As an example a partial creation of a SPARQL implementing  the z-score normalization function is presented in Figure~\ref{fig:sparql-zscore}.
 \item In order to promote new observations to the different components and indexes 
 a set of raw observations is also loaded and a new AST walker generates new values, through SPARQL queries (see Figure~\ref{fig:sparql-generated-query}), 
 in a bottom-up approach until reaching the upper-level (index), see Table~\ref{tab:example-generated-wb} and Figure~\ref{fig:generated observation}. 
 Finally each new observation also contains a data property to store the hash MD5 of a string comprising 
 (publisher, $q$, $D$, $v$) and separated by \#, e.g. ``RDFIndex\#TheWorldBankNaiveIndex\#Spain\#2010\#0.707'', to avoid potential ``man-in-the-middle'' attacks when the new observations 
 are published. This value are easily generated using the new string functions in SPARQL 1.1 \texttt{concat} and \texttt{MD5}.
 \end{itemize}

% \begin{figure}[!ht]
% \begin{lstlisting}[language=SQL]  
% prefix afn: <http://jena.hpl.hp.com/ARQ/function#>
% SELECT ( (?measure-?mean)/?stddev as ?zscore) 
% WHERE{
%  ...
%  ?observation rdfindex:value ?measure 
%  {
%   SELECT ?mean (afn:sqrt((SUM((?measure-?mean)*(?measure-?mean))/?count)) 
% 		as ?stddev) 
%   WHERE{ 
%    ?observation rdfindex:value ?measure 
%    {
%     SELECT (COUNT(?measure) as ?count) (AVG(?measure) as ?mean)
%     WHERE {
%      ?observation rdfindex:value ?measure 
%      }GROUP BY ?count ?mean LIMIT 1
% 
%    }	
%   }GROUP BY ?mean ?count LIMIT 1
%  }
% }
% \end{lstlisting}
% \caption{Z-score normalization using SPARQL.}
%  \label{fig:sparql-zscore}
% \end{figure}


% 
\begin{figure}[!ht]
\begin{lstlisting}[language=SQL,basicstyle=\scriptsize]  
prefix rdfindex:  <http://purl.org/rdfindex/ontology/> 
SELECT ?dim0 ?dim1 ( sum(?w*?measure) as ?newvalue) 
WHERE{ 
  rdfindex-wb:TheWorldBankNaiveIndex  
  rdfindex:aggregates ?parts.
  ?parts rdfindex:part-of ?partof.
  ?partof rdfindex:dataset ?part .
  FILTER ((?part =rdfindex-wb:AidEffectiveness) || 
	  (?part =rdfindex-wb:Health)). 
  ?observation qb:dataSet ?part . 
  ?part rdfindex:weight ?defaultw.        
  OPTIONAL {?partof rdfindex:weight ?aggregationw.}.
  BIND (if( BOUND(?aggregationw), ?aggregationw, ?defaultw) AS ?w)
  ?observation rdfindex:value ?measure . 
  ?observation rdfindex-wbont:ref-area ?dim0. 
  ?observation rdfindex-wbont:ref-year ?dim1. 
} GROUP BY ?dim0 ?dim1 
\end{lstlisting}
\caption{Example of generated SPARQL query.}
 \label{fig:sparql-generated-query}
\end{figure}



\begin{table}[!htb]
\renewcommand{\arraystretch}{1.3}
\scriptsize
\begin{center}
\begin{tabular}{|p{4.5cm}|p{1cm}|p{1.5cm}|p{1cm}|p{1.8cm}|}
\hline
  \textbf{Description} & \textbf{Year} & \textbf{Country} & \textbf{Value} & \textbf{Status} \\  \hline
  \multicolumn{5}{|c|}{\textbf{Indicator}} \\ \hline
  Aggregated Life Expectancy  & 2010 & Spain & $82$ & Estimated \\ \hline
  Aggregated Life Expectancy  & 2011 & Spain & $82$ & Estimated \\ \hline
  Aggregated Life Expectancy  & 2010 & Greece & $80.5$ & Estimated \\ \hline
  Aggregated Life Expectancy  & 2011 & Greece & $81$ & Estimated \\ \hline
  \multicolumn{5}{|c|}{\textbf{Component}} \\ \hline
  Aid Efectiveness  & 2010 & Spain & $0.707$  & Estimated \\ \hline
  Aid Efectiveness  & 2011 & Spain & $0.707$  & Estimated \\ \hline
  Aid Efectiveness  & 2010 & Greece & $-0.707$ & Estimated \\ \hline
  Aid Efectiveness  & 2011 & Greece & $-0.707$ & Estimated \\ \hline
  Health & 2010 & Spain & $0.707$ & Estimated \\ \hline
  Health & 2011 & Spain & $0.707$ & Estimated \\ \hline
  Health & 2010 & Greece & $-0.707$ & Estimated \\ \hline
  Health & 2011 & Greece & $-0.707$ & Estimated \\ \hline
 \multicolumn{5}{|c|}{\textbf{Index}} \\ \hline
  The Naive World Bank Index & 2010 & Spain & $0.707$ & Estimated \\ \hline
  The Naive World Bank Index & 2011 & Spain & $0.707$ & Estimated \\ \hline
  The Naive World Bank Index & 2010 & Greece & $-0.707$ & Estimated \\ \hline
  The Naive World Bank Index & 2011 & Greece & $-0.707$ & Estimated\\ \hline
  \hline
  \end{tabular}
  \caption{Example of aggregated observations from the WorldBank.}
  \label{tab:example-generated-wb}
  \end{center}	 
\end{table} 



\begin{figure}[!ht]
\begin{lstlisting}[language=SQL,basicstyle=\scriptsize]  
rdfindex-wb:o6808100851579
      a       qb:observation ;
      qb:dataSet rdfindex-wb:TheWorldBankNaiveIndex  ;
      rdfindex-wbont:ref-area dbpedia-res:Spain ;
      rdfindex-wbont:ref-year
	<http://reference.data.gov.uk/id/
	gregorian-interval/2010-01-01T00:00:00/P1Y> ;
      ...
      #rdfs:{label,comment} {literal};
      #dcterms:{issued, date, contributor, author, publisher, identifier} 
      #	{resource, literal};
      rdfindex:md5-hash "6cdda76088cd161d766809d6a78d35f6";
      sdmx-concept:obsStatus
              sdmx-code:obsStatus-E;
      rdfindex:value "0.707"^^xsd:double .
\end{lstlisting}
\caption{Partial example of a populated observation for ``The World Bank Naive Index''.}
 \label{fig:generated observation}
\end{figure}


% FIXME: A Java code snippet of visitors?

% \section{FIXME: Use Case: TheCloudindex (not in this version)}
% % % 
\clearpage
\section{Discussion and Further Steps}
Previous sections have presented the motivation of this work, a vocabulary designed on top of the 
RDF Data Cube Vocabulary for modeling quantitative indexes and a Java-SPARQL based processor 
to validate and compute any kind of index. According to the requirements and features 
of quantitative indexes an initial evaluation is presented in Table~\ref{tab:eval-rdfindex} and Table~\ref{tab:eval-rdfindex-cross} 
to show how the application of semantic technologies can ease indexes management and their computation process 
to make the most of data~\footnote{http://fcw.com/articles/2013/06/11/data-performance-management.aspx}: 
collect, verify, analyze, apply, share, protect, archive and re-use.

Furthermore it is relevant to remark that data quality and filtering is currently a big challenge 
due to the vast amount of data that is continuously being generated. Policy-makers as agents in charge 
of making decisions must be able to manage this information in a timely and flexible fashion. In this context semantic technologies provides the adequate and standardized building blocks to improve the dynamism and updating capabilities of policy-maker tools. Nevertheless 
the initial effort to translate existing index definitions and computation processes to this approach can 
be hard and time-consuming but going beyond of that further updates and tools can perfectly benefit from this effort as 
other ``semantized'' domains have already done. 

 
\begin{table}[!htb]
\renewcommand{\arraystretch}{1.3}
\begin{center}
\begin{tabular}{|p{3cm}|p{3.5cm}|p{6.5cm}|}
\hline
  \textbf{Feature} & \textbf{Crucial Step} &\textbf{Main advantages}  \\  \hline
  Data sources &  \begin{itemize} \item Collect. \item Verify. \item Tag (enrichment).\end{itemize} & \begin{itemize}
		      \item Common and shared data model, RDF.
		      \item Description of data providers (provenance and trust).
		      \item A formal query language to query data, SPARQL.
		      \item Use of Internet protocols, HTTP.
		      \item Data enrichment and validation (domain and range).
		      \item Unique identification of entities, concepts, etc. through (HTTP) URIs.
		      \item Possibility of publishing new data under the aforementioned characteristics.
		      \item Standardization and integration of data sources.
		    \end{itemize} \\ \hline  
  Structure & \begin{itemize} \item Verify. \item Tag (enrichment). \item Analyze. \item Share. \item Archive. \item Re-use. \end{itemize} & \begin{itemize}
                  \item Meta-description of index structure (validation).
                  \item Re-use of existing semantic web vocabularies.
                  \item Re-use of existing datasets to enrich meta-data.
                  \item Context-aware definitions.
                  \item Underlying logic formalism.                  
                  \item Orthogonal and flexible.                 
                 \end{itemize} \\ \hline  
  Computation process &  \begin{itemize} \item Verify. \item Tag (enrichment). \item Apply. \item Share. \item Archive. \item Re-use. \end{itemize} & \begin{itemize}
                  \item Meta-description of datasets aggregation.
                  \item Validation of composed datasets.
                  \item OWA operators support.
                  \item Direct translation to SPARQL queries.  
                 \end{itemize} \\ \hline
  Documentation & \begin{itemize} \item Verify. \item Apply. \item Share. \item Archive. \item Re-use. \end{itemize} &\begin{itemize}
                  \item Multilingual support to describe datasets, etc.
                  \item Easy generation with existing tools.                
                 \end{itemize} \\ \hline               
       
  \hline
  \end{tabular}
  \caption{Initial evaluation of semantic technologies for modeling and computing quantitative indexes.}
  \label{tab:eval-rdfindex}
  \end{center}
\end{table} 



\begin{table}[!htb]
\renewcommand{\arraystretch}{1.3}
\begin{center}
\begin{tabular}{|p{3.5cm}|p{3cm}|p{6.5cm}|}
\hline
  \textbf{Feature} & \textbf{Crucial Step} &\textbf{Main advantages}  \\  \hline
   Cross-Domain Features & All & \begin{itemize}
                  \item Separation of concerns and responsibilities: data and meta-data (structure and computation).
                  \item Standardization (put in action specs from organisms such as W3C).
                  \item Declarative and adaptive approach.
                  \item Non-vendor lock-in (format, access and computation).
                  \item Integration and Interoperability.
                  \item Transparency.
                  \item Help to build own indexes.
                  \item Align to existing trend (data management: quality and filtering)
                  \item Easy integration with third-party services such as visualization.
                  \item Contribution to the Web of Data.
                 \end{itemize} \\ \hline              
  \hline
  \end{tabular}
  \caption{Cross-Domain initial evaluation of semantic technologies for modeling and computing quantitative indexes.}
  \label{tab:eval-rdfindex-cross}
  \end{center}
\end{table} 


 
\section{Conclusions and Future Work}
Data filtering, quality and aggregation is a major challenge in the 
new data-driven economy. The interest of public and private bodies 
to manage and exploit information is growing and specific activities 
such as policy making must take advantage of compiling data and information 
to make more timely and accurate decisions. In this sense the use 
of quantitative indexes has been widely accepted as a practice to 
summarize diverse key performance indicators and establish a numerical 
value with the aim of comparing different alternatives. Nevertheless traditional techniques 
and the ``infobesity'' are preventing a proper use of information and the dynamism 
of this new environment also requires a high-level of integration and interoperability. In this 
context the Semantic Web area and, more specifically the Linked Data initiative, 
provide a standardized stack of technologies for knowledge-based techniques with 
the aim of boosting data management and exploitation. That is why the application 
of semantic web principles to model quantitative indexes from both structural and 
computational points of view can fulfill the requirements of this new data-based 
environment and leverage semantics to create a real knowledge-based society. The 
present work has introduced two main contributions: 1) the RDFIndex vocabulary to model and compute quantitative indexes and 
2) a Java-SPARQL based processor of the RDFIndex to validate, generate and populate new observations.

Although this first effort has applied semantics to a specific problem new enhancements 
must be done to cover all potential requirements in the construction of quantitative indexes. In this sense, the presented 
approach is now being used in the modeling of indexes such as the Cloud Index~\footnote{\url{http://cloudindex-doc.herokuapp.com}} or 
the Web Index~\cite{webindexlod}. Furthermore, the proposed representation contains a very interesting property 
that lies in the possibility of computing indexes and components in a parallel fashion. 
Other issues such as SPARQL-based validation, provenance and trust, real-time updates, storage or visualization are 
also on-going work that must be addressed to fit to new data environments.







\section{Acknowledgements}
This work is part of the FP7 Marie Curie Initial Training Network ``RELATE'' (cod. 264840) and developed in the context 
of the Workpackage 4 and more specifically under the project ``Quality Management in Service-based Systems and Cloud Applications''. It is 
also funded by the ROCAS project with code TIN2011-27871, a research project partially funded by the Spanish Ministry of Science and Innovation.

\clearpage

\bibliographystyle{plain}
% %\bibliographystyle{unsrt}
% % %\bibliographystyle{acm}
\bibliography{references}
 % \renewcommand{\bibname}{References}

\end{document}

