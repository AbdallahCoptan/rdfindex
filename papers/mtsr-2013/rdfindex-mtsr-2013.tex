\documentclass{llncs}

%\usepackage{llncsdoc}

%\usepackage{makeidx}  % allows for indexgeneration
\usepackage{graphicx}
\usepackage[T1]{fontenc}
\usepackage[english]{babel}
\usepackage[utf8]{inputenc}
\usepackage{multirow}

\usepackage{url}
\usepackage{rotating}

%%%Math
\usepackage{latexsym}
% \usepackage{amsmath}
% \usepackage{amssymb}
% \usepackage{amsthm}
%\usepackage{eurosans}

\usepackage{eurosym}

\usepackage{longtable}

\usepackage{listings}

\usepackage{color}
\usepackage{textcomp}


\definecolor{gray}{gray}{0.5}
\definecolor{green}{rgb}{0,0.5,0}


\begin{document}
\title{Leveraging semantics to represent and compute quantitative indexes. \\ The RDFindex approach.}

\author{Jose Mar\'{i}a \'{A}lvarez\inst{1} \and Jos\'{e} Emilio Labra\inst{3}}


\authorrunning{Jose Mar\'{i}a Alvarez \and Michail Vafolopoulos \and  Jos\'{e} Emilio Labra}


\institute{South East European Research Center, \\ 54622, Thessaloniki, Greece\\
  \email{{jmalvarez@seerc.org}}\\
\and WESO Research Group, Department of Computer Science, University of Oviedo, \\ 33007, Oviedo, Spain.\\
  \email{{labra@uniovi.es.es}}\\  
}


\date{}

\maketitle

\renewcommand{\labelitemi}{$\bullet$}

\begin{abstract}
FIXME
\end{abstract}

\section{Introduction}


As a motivating example, see Table~\ref{tab:example-wb}, a policy-maker wants to re-use the WorldBank data to model and compute a new index, ``The Naive World Bank Index''. This 
index uses the topics ``Aid Efectiveness'' ($c_1$) and ``Health'' ($c_2$) with the aim of comparing the status and evolution of health in several countries to decide whether new 
investments are performed. From these components two indicators has been respectively selected by experts: ``Life Expectancy'' ($i_1$) and ``Health expenditure, total (\%) of GDP'' ($i_2$). 
Once components and their indicators are defined and data can be retrieved from the WorldBank it is necessary to set how the index and components are computed 
taking into account that only data about indicators is available. Following a top-down approach the index, $i$, is calculated through ordered weighted averaging (OWA) using the 
formula: $\sum_{i=1}^n  w_i c_i$, where $w_i$ is the weight of the component $c_i$. On the other hand, both components only aggregates one indicator but the ``Aid Efectiveness'' 
must firstly compute the ``Life Expectancy'' without considering the sex dimension. Making this implies the need of calculating the average age by country and year to create 
a new ``derivated'' indicator of ``Life Expectancy''. Apart from that the computation must also consider the observation status and values must be normalized using the 
z-score before computing intermediate and final values for each indicator, component and index. Furthermore this index is supposed to change in the future 
adding new data sources, modifying the computation processes (weights) or the structure (new components, indicators and dimensions). Finally, the policy-maker 
is also interested in applying this index to other scenarios and he needs a way of explaining in different languages how the index is computed. 


\begin{table}[!htb]
\renewcommand{\arraystretch}{1.3}
\begin{center}
\begin{tabular}{|p{3cm}||p{5cm}|p{1.8cm}|p{1.8cm}|p{1cm}|}
\hline
  \textbf{Component} & \textbf{Indicator} & \textbf{Year} & \textbf{Country} & \textbf{Value}  \\  \hline
  Aid Efectiveness & Life Expectancy Male & 2010 & Spain & $1.0$ \\ \hline
  Aid Efectiveness & Life Expectancy Male & 2011 & Spain & $1.0$ \\ \hline
  Aid Efectiveness & Life Expectancy Female & 2010 & Spain & $1.0$ \\ \hline
  Aid Efectiveness & Life Expectancy Female & 2011 & Spain & $1.0$ \\ \hline
  Aid Efectiveness & Life Expectancy Male & 2010 & Greece & $1.0$ \\ \hline
  Aid Efectiveness & Life Expectancy Male & 2011 & Greece & $1.0$ \\ \hline
  Aid Efectiveness & Life Expectancy Female & 2010 & Greece & $1.0$ \\ \hline
  Aid Efectiveness & Life Expectancy Female & 2011 & Greece & $1.0$ \\ \hline
  Health & Health expenditure, total (\%) of GDP) & 2010 & Spain & $1.0$ \\ \hline
  Health & Health expenditure, total (\%) of GDP) & 2011 & Spain & $1.0$ \\ \hline
  Health & Health expenditure, total (\%) of GDP) & 2010 & Spain & $1.0$ \\ \hline
  Health & Health expenditure, total (\%) of GDP) & 2011 & Spain & $1.0$ \\ \hline
  \hline
  \end{tabular}
  \caption{Example of observations from the WorldBank.}
  \label{tab:example-wb}
  \end{center}
\end{table} 


\section{Related Work}
\section{Theoretical modelling of a quantitative composite index}
\section{Representation of a quantitative composite index in RDF: The RDFIndex}
\section{A $\{Java\rightarrow based \rightarrow SPARQL\}$  interpreter of the RDFIndex}
%
\subsection{Use Case: FIXME}
% % 
\section{Discussion and Further Steps}

\section{Conclusions and Future Work}

\end{document}

