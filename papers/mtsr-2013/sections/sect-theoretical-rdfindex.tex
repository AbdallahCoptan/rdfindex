This section outlines a model for representing quantitative indexes based on the aggregation 
of different components and indicators. Furthermore a computation process for those elements is 
also presented in order to specify the population of new observations.

Basically, a quantitative index is comprised of the aggregation of several components. In the same way, 
a component is also composed of the aggregation of indicators that keep real observations. From this initial definition 
some characteristics and assumptions can be found: 1) although observations can be directly mapped to an index or a component, they 
are usually computed applying a bottom-up approach from an indicator to a component and so on. 2) An observation is 
a real numerical value extracted from some agent out under a certain context. Generally observations only takes one measure and are considered 
to be raw without any pre-processing technique. 3) Before aggregating observation values, componens and indexes can 
estimate missing values to finally normalize them in order to get a quantitative value.

According to the aforementioned characteristics and assumptions an ``observable'' element (index, component or indicator) is a 
dataset of numerical observations under a specific context (dimensions and/or meta-data) that can be directly extracted from external 
sources out or computed by some kind of OWA operator. 

\begin{definition}[Observation-$o$]\upshape
It is a tuple $\{v,m,s\}$, where $v$ is a numerical value for the measure $m$ with an status $s$ that belongs to 
only one dataset of observations $O$. 
\end{definition}


\begin{definition}[Dataset-$q$]\upshape
It is a tuple $\{O,m,D,A,T\}$ where $O$ is a set of observations for only one measure $m$ that is described under 
a set of dimensions $D$ and a set of annotations $A$. Additionally, some attributes can be defined in the set $T$ for structure enrichment. 
\end{definition}


\begin{definition}[Aggregated dataset-$aq$]\upshape
It is an aggregation of $n$ datasets $q_i$ (identified by the set $Q$) which set of observations $O$ is derivated by applying 
an OWA operator $p$ to the observations $O_{q_i}$. 
\end{definition}

As a necessary condition for the computation process, an aggregated dataset $aq$ defined by means of the set of dimensions $D_{aq}$ can be computed iif 
$\forall q_j \in Q: D_{aq} \subseteq D_{q_j}$. Furthermore the OWA operator $p$ can only aggregate values belonging to the same measure $m$.

As a consequence of the aforementioned definitions some remarks must be outlined in order to restrict the understanding of 
a quantitative index (structure and computation):
\begin{itemize}
 \item The set of dimensions $D$, annotations $A$ and attributes $T$ for a given dataset $Q$ is always the same with the aim of describing all observations under 
 the same context.
 \item An index $i$ and a component $c$ are aggregated datasets. Nevertheless this restriction is relaxed if observations can be directly mapped to 
 these elements without any computation process.
 \item An indicator $in$ can be both dataset or aggregated dataset.
 \item All elements in definitions must be uniquely identified. 
 \item An aggregated dataset is also a dataset. FIXME: demonstration, relational algebra?
\end{itemize}

Following the on-going example, see Table~\ref{tab:example-wb}, the modeling of the ``The Naive World Bank Index'' would be the next one:
\begin{itemize}
 \item Each row of the table is an observation $o_i$ with a numerical value $v$, the measure is $m_{in}$ and the status is ``Raw''.
 \item Two indicators can be found: \{ ($in_1$, ``Life Expectancy''), ($in_2$, ``Health expenditure, total (\% of GDP)'') \}, each indicator contains a set 
 of observations $O_{in_i}$. The dimensions for each indicator are: $D_{in_1}$  \{(``Year'', ``Country'', ``Sex''\} and $D_{in_2}$ \{``Year'', ``Country''\}.
 \item In order to group the ``Life Expectancy'' without the ``Sex'' dimension it is necessary to define a new aggregated dataset $aq_1$ which 
 dimensions $D_{aq_1}$ are \{``Year'', ``Country''\} and the OWA operator is the average of values $v \in O_{in_1}$. In this sample the aggregated indicator $aq_1$
 can be built due to the indicator ``Life Expectancy'' accomplishes with the aforementioned necessary conditions: 1) $D_{aq} \subseteq D_{in_1} \wedge D_{aq_1} \subseteq D_{in_2}$ and 
 2) $m_{aq_1}= m_{in_1}$ = ``Life Expectancy''.
 \item In the same way, the set of components: \{($c_1$,``Aid Efectiveness''), ($c_2$,``Health'')\} are built aggregating the indicators $aq_1$ and 
 $q_2$ using as OWA operator the ``min'' value. In this case ``min'' or ``max'' operators can be used due to an observation is uniquely identified in a 
 dataset by a tuple $\{v,m,s\} \cup D$.
 \item Finally, the index is computed using the general form of an OWA operator $\sum_{i=1}^n  w_i c_i$ and taking as weights those we select.
\end{itemize}

As final remark, the computation process is generating new observations, following a bottom-up approach, according to the structure defined 
in each dataset. Although a logical structure of indexes, components and indicators can be directly established using narrower/broader properties 
the main advantage of this approach lies in the possibility of expressing new elements by aggregating others describing their structure. Nevertheless restrictions 
about the type of dataset that can be aggregated in each level could be added at any time for other reasons such as validation or to generate 
a human-readable form of the index.


\section{Representation of a quantitative composite index in RDF: The RDFIndex}
Since previous section has stated the building blocks to represent quantitative indexes by aggregation a direct translation built 
on top of the RDF Data Cube Vocabulary, SDMX and other semantic web vocabularies is presented in Table~\ref{index-to-rdf}. Thus 
all concepts in the index are described reusing existing definitions, taking advantage of previous efforts and pre-established semantics 
with the aim of being extended in the future to fit new requirements. According to these mappings a definition of the index in the 
on-going example and some dimensions are presented in Figure~\ref{fig:results-rdf-index} and Figure~\ref{fig:results-rdf-properties}.

FIXME: Maybe an observation? include a table with all reused vocabularies? a diagraa of the ontology?

\begin{table}[!htb]
\renewcommand{\arraystretch}{1.3}
\begin{center}
\begin{tabular}{|p{3cm}||p{6cm}|p{3cm}|}
\hline
  \textbf{Concept} & \textbf{Vocabulary element} &  \textbf{Comments}  \\  \hline
   Observation $o$ & \texttt{qb:Observation} &  \\ \hline
   Numerical value $v$ & \texttt{xsd:double} &  \\ \hline
   Measure $m$ & \texttt{qb:MeasureProperty} \texttt{sdmx-measure:obsValue} &  \\ \hline
   Status $s$ & \texttt{sdmx-concept:obsStatus} &  \\ \hline
   Dataset $q$ & \texttt{qb:dataSet} and \texttt{qb:qb:DataStructureDefinition} &  \\ \hline
   Dimension $d_i \in D$ & \texttt{qb:DimensionProperty} &  \\ \hline
   Annotation $a_i \in A$ & \texttt{owl:AnnotationProperty} &  FIXME: dublin core?\\ \hline
   Attribute $at_i \in T$ & \texttt{qb:AttributeProperty} &  \\ \hline
   OWA operator $p$ &  \texttt{skos:Concept} and SPARQL 1.1 aggregation operators & FIXME \\ \hline
   Index, Component and Indicator & \texttt{skos:Concept} & FIXME: structure \\ \hline
  \hline
  \end{tabular}
  \caption{Mapping between the index definition and the RDF Data Cube Vocabulary.}
  \label{index-to-rdf}
  \end{center}
\end{table} 

\begin{figure}[!ht]
\begin{lstlisting}[language=XML]  
@prefix rdfindex:  <http://purl.org/rdfindex/ontology/> .
@prefix rdfindex-wb:  <http://purl.org/rdfindex/wb/resource/> .
@prefix rdfindex-wbont:  <http://purl.org/rdfindex/wb/ontology/> .

rdfindex-wb:TheWorldBankNaiveIndex 
  a rdfindex:Index;
  rdfs:label "The Weight Longest Life Country"@en;
  rdfindex:type rdfindex:Quantitative;
  rdfindex:aggregates [ 		
    rdfindex:aggregation-operator rdfindex:OWA;
    rdfindex:part-of [
      rdfindex:dataset rdfindex-wb:AidEffectiveness; 
      rdfindex:weight 0.4];
    rdfindex:part-of [rdfindex:dataset rdfindex-wb:Health; 
      rdfindex:weight 0.6];
  ];
  #More meta-data properties...
  qb:structure 	rdfindex-wb:TheWorldBankNaiveIndexDSD ; .
  
rdfindex-wb:TheWorldBankNaiveIndexDSD 
  a qb:DataStructureDefinition;  
  qb:component    
  [qb:dimension rdfindex-wbont:ref-area],
  [qb:dimension rdfindex-wbont:ref-year],
  [qb:measure   rdfindex:value],
  [qb:attribute sdmx-attribute:unitMeasure];
  #More meta-data properties...
  .
\end{lstlisting}
\caption{Example of an index structure in RDF.}
 \label{fig:results-rdf-index}
\end{figure}

\begin{figure}[!ht]
\begin{lstlisting}[language=SQL]  
rdfindex-wbont:ref-area a rdf:Property, 
  qb:DimensionProperty; 
   rdfs:subPropertyOf sdmx-dimension:ref-area; 
  rdfs:range skos:Concept; 
  qb:concept sdmx-concept:ref-area . 

rdfindex:value a rdf:Property, qb:MeasureProperty;
  rdfs:label "Value of an observation"@en;
  skos:notation "value" ;
  rdfs:subPropertyOf sdmx-measure:obsValue;
  rdfs:range xsd:double . 
\end{lstlisting}
\caption{Example of a dimension and a measure definition in RDF.}
 \label{fig:results-rdf-properties}
\end{figure}


Once the structure and the computation processes can be built on the top of existing RDF vocabulary it is also 
possible to make a translation to a generic SPARQL query (includes the basic OWA operator), see Figure~\ref{fig:results-rdf-sparql-template}, 
in order generate new observations following the bottom-up approach that previous section has presented.

\begin{figure}[!ht]
\begin{lstlisting}[language=SQL,mathescape]  
SELECT ($d_i \in D$) [(sum(?w*?measure) as ?newvalue) | OWA(?measure)]
WHERE{
  $q$ rdfindex:aggregates ?parts.
  ?parts rdfindex:part-of ?partof.
  ?partof rdfindex:dataset $q_i$ .
  FILTER($?partof \in Q$).  
  ?observation rdf:type qb:Observation.
  ?part rdfindex:weight ?defaultw.     
  OPTIONAL {?partof rdfindex:weight ?aggregationw.}.
  BIND (if( BOUND(?aggregationw), ?aggregationw, ?defaultw) AS ?w)
  ?observation $m$ ?measure . 
  ?observation ?dim ?dimRef. 
  FILTER ($?dim  \in D$).
}GROUP BY ($d_i \in D$)
\end{lstlisting}
\caption{SPARQL template for building aggregated observations.}
 \label{fig:results-rdf-sparql-template}
\end{figure}


\subsection{A Java-SPARQL based interpreter of the RDFIndex}
The first implementation of the RDFIndex vocabulary processor is based on traditional 
language processor techniques such as the use of design patterns (e.g. Composite or Visitor) 
to separate the exploitation of meta-data from the interpretation. Thus the processor 
works and provides next functionalities (hereafter load and query an endpoint are completely equivalent due to 
data is separated from access and storage formats):

\begin{itemize}
 \item The RDFIndex ontology is loaded to have access to common definitions.
 \item The structure of an index described with the aforementioned vocabulary is 
 also loaded to create a kind of Abstract Syntax Tree (AST) containing the defined meta-data.
 \item Once the meta-data structure is available in the AST it can be managed 
 through three AST walkers that performs: 1) validation (structure and RDF Data Cube normalization) 
 and 2) SPARQL queries creation and 3) documentation generation (optional). As an example a partial creation of a SPARQL implementing 
 the z-score normalization function is presented in Figure~\ref{fig:sparql-zscore}.
 \item In order to promote new observations to the different components and indexes 
 a set of raw observations is also loaded and a new AST walker generates new values, through SPARQL queries (see Figure~\ref{fig:sparql-generated-query}), 
 in a bottom-up approach until reaching the upper-level (index), see Table~\ref{tab:example-generated-wb} and Figure~\ref{fig:generated observation}. Finally 
 each new observation also contains a data property to store the hash MD5 of a string comprising 
 (publisher, $q$, $D$, $v$) and separated by \#, e.g. ``RDFIndex\#TheWorldBankNaiveIndex\#Spain\#2010\#0.707'', to avoid ``man-in-the-middle'' attacks when the new observations 
 are published.
 \end{itemize}

\begin{figure}[!ht]
\begin{lstlisting}[language=SQL]  
prefix afn: <http://jena.hpl.hp.com/ARQ/function#>
SELECT ( (?measure-?mean)/?stddev as ?zscore) 
WHERE{
 ...
 ?observation rdfindex:value ?measure 
 {
  SELECT ?mean (afn:sqrt((SUM((?measure-?mean)*(?measure-?mean))/?count)) 
		as ?stddev) 
  WHERE{ 
   ?observation rdfindex:value ?measure 
   {
    SELECT (COUNT(?measure) as ?count) (AVG(?measure) as ?mean)
    WHERE {
     ?observation rdfindex:value ?measure 
     }GROUP BY ?count ?mean LIMIT 1

   }	
  }GROUP BY ?mean ?count LIMIT 1
 }
}
\end{lstlisting}
\caption{Z-score normalization using SPARQL.}
 \label{fig:sparql-zscore}
\end{figure}


% 
\begin{figure}[!ht]
\begin{lstlisting}[language=SQL]  
prefix rdfindex:  <http://purl.org/rdfindex/ontology/> 
SELECT ?dim0 ?dim1 ( sum(?w*?measure) as ?newvalue) 
WHERE{ 
  rdfindex-wb:TheWorldBankNaiveIndex  
  rdfindex:aggregates ?parts.
  ?parts rdfindex:part-of ?partof.
  ?partof rdfindex:dataset ?part .
  FILTER ((?part =rdfindex-wb:AidEffectiveness) || 
	  (?part =rdfindex-wb:Health)). 
  ?observation qb:dataSet ?part . 
  ?part rdfindex:weight ?defaultw.        
  OPTIONAL {?partof rdfindex:weight ?aggregationw.}.
  BIND (if( BOUND(?aggregationw), ?aggregationw, ?defaultw) AS ?w)
  ?observation rdfindex:value ?measure . 
  ?observation rdfindex-wbont:ref-area ?dim0. 
  ?observation rdfindex-wbont:ref-year ?dim1. 
} GROUP BY ?dim0 ?dim1 
\end{lstlisting}
\caption{Example of generated SPARQL query.}
 \label{fig:sparql-generated-query}
\end{figure}



\begin{table}[!htb]
\renewcommand{\arraystretch}{1.3}
\begin{center}
\begin{tabular}{|p{4.5cm}|p{1cm}|p{1.5cm}|p{1cm}|p{1.8cm}|}
\hline
  \textbf{Description} & \textbf{Year} & \textbf{Country} & \textbf{Value} & \textbf{Status} \\  \hline
  \multicolumn{5}{|c|}{\textbf{Indicator}} \\ \hline
  Aggregated Life Expectancy  & 2010 & Spain & $82$ & Estimated \\ \hline
  Aggregated Life Expectancy  & 2011 & Spain & $82$ & Estimated \\ \hline
  Aggregated Life Expectancy  & 2010 & Greece & $80.5$ & Estimated \\ \hline
  Aggregated Life Expectancy  & 2011 & Greece & $81$ & Estimated \\ \hline
  \multicolumn{5}{|c|}{\textbf{Component}} \\ \hline
  Aid Efectiveness  & 2010 & Spain & $0.707$  & Estimated \\ \hline
  Aid Efectiveness  & 2011 & Spain & $0.707$  & Estimated \\ \hline
  Aid Efectiveness  & 2010 & Greece & $-0.707$ & Estimated \\ \hline
  Aid Efectiveness  & 2011 & Greece & $-0.707$ & Estimated \\ \hline
  Health & 2010 & Spain & $0.707$ & Estimated \\ \hline
  Health & 2011 & Spain & $0.707$ & Estimated \\ \hline
  Health & 2010 & Greece & $-0.707$ & Estimated \\ \hline
  Health & 2011 & Greece & $-0.707$ & Estimated \\ \hline
 \multicolumn{5}{|c|}{\textbf{Index}} \\ \hline
  The Naive World Bank Index & 2010 & Spain & $0.707$ & Estimated \\ \hline
  The Naive World Bank Index & 2011 & Spain & $0.707$ & Estimated \\ \hline
  The Naive World Bank Index & 2010 & Greece & $-0.707$ & Estimated \\ \hline
  The Naive World Bank Index & 2011 & Greece & $-0.707$ & Estimated\\ \hline
  \hline
  \end{tabular}
  \caption{Example of aggregated observations from the WorldBank.}
  \label{tab:example-generated-wb}
  \end{center}	 
\end{table} 



\begin{figure}[!ht]
\begin{lstlisting}[language=SQL]  
rdfindex-wb:o6808100851579
      a       qb:observation ;
      qb:dataSet rdfindex-wb:TheWorldBankNaiveIndex  ;
      rdfindex-wbont:ref-area dbpedia-res:Spain ;
      rdfindex-wbont:ref-year
	<http://reference.data.gov.uk/id/
	gregorian-interval/2010-01-01T00:00:00/P1Y> ;
      ...
      #rdfs:{label,comment} {literal};
      #dcterms:{issued, date, contributor, author, publisher, identifier} 
      #	{resource, literal};
      rdfindex:md5-hash "6cdda76088cd161d766809d6a78d35f6";
      sdmx-concept:obsStatus
              sdmx-code:obsStatus-E;
      rdfindex:value "0.707"^^xsd:double .
\end{lstlisting}
\caption{Partial example of populated observation for ``The World Bank Naive Index''.}
 \label{fig:generated observation}
\end{figure}


FIXME: A Java code snippet of visitors?
