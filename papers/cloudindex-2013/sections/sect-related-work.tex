%Open Data
Recent times have seen the deployment of the Open Data initiative due to the strategy 
developed by the governments in USA, UK or Australia and that have been followed by most of countries around the world to deliver data 
catalogues containing valuable public sector information (PSI). This is not a new concept but a recent formalized definition made 
by the Open Definition can be summarized as: ``A piece of data is open if anyone is free to use, re-use, and redistribute it — subject only, at most, to the requirement 
to attribute and/or share-alike.'' In this context the Open (Government) Data movement (W3C) is making a great effort 
to spread this view in public and private bodies with the objective of boosting transparency and a new data-driven economy. 
From a corporate strategy point of view the re-use of existing open data must encourage and improve 
more efficient policy-making processes. Under this view there is a perfect matching between the Linked Data and the Open Data 
principles: on the one hand the Linked Data community is generating the proper environment of standard 
specifications and tools to manage data and, on the other hand, the Open Data initiative is requiring 
the right methods to generate a real re-use of the public data. The combination of these 
two views leads us to the Linked Open Data (LOD) effort that seeks for applying the principles of 
Linked Data to implement the Open Data mission.

Currently one of the mainstreams in the Semantic Web area lies in the application of the LOD initiative in 
different domains such as  e-Government, e-Health, Biomedicine, Education, Bibliography or Geography to name a few,  
with the aim of solving existing problems of integration and interoperability among applications and create a 
knowledge environment under the Web-based protocols. In this context, the present work is therefore focused 
in applying semantic web vocabularies and datasets to model quantitative indexes from both structural 
and computational points of view in a ``Policy-Making'' context. 

In order to reach the major objective of building a re-usable Web of Data,  
the publication of information and data under a common data model (RDF) and formats with 
a specific formal query language (SPARQL~\cite{Sparql11}) provide the required building blocks to turn 
the Web of documents into a real database of data~\cite{freebase}. As a consequence the popular 
diagram of the LOD Cloud~\cite{linked-data-cloud}, generated from meta-data extracted from the Comprehensive Knowledge Archive Network (CKAN~\cite{ckan}) out, 
contains $337$ datasets, with more than $25$ billion RDF triples and $395$ million links in different  domains. 
Research works are focused in two main areas: 1) production/publishing~\cite{bizer07how} and 2) consumption. 
In the first case data quality~\cite{bizer2007,wiqa,ld-quality,lodq,link-qa}, conformance~\cite{HoganUHCPD:2012:237}, 
provenance~\cite{w3c-prov,DBLP:conf/ipaw/HartigZ10}, trust~\cite{Carroll05namedgraphs}, description of datasets~\cite{void,Cyganiak08semanticsitemaps,ckanValidator} and 
entity reconciliation~\cite{Serimi,Maali_Cyganiak_2011} are becoming major objectives since a mass of amount data is already available~\cite{Triplify} 
through SPARQL endpoints deployed on the top of RDF repositories such as OpenLink Virtuoso or OWLim. 

On the other hand, consumption of Linked Data is being addressed to provide new ways of data visualization~\cite{DBLP:journals/semweb/DadzieR11,hoga-etal-2011-swse-JWS}, 
faceted browsing~\cite{Pietriga06fresnel,citeulike:8529753,Sparallax} and searching~\cite{hoga-etal-2011-swse-JWS}, processing~\cite{Harth:2011:SIP:1963192.1963318} and exploitation of data applying 
different approaches such as sensors~\cite{Jeung:2010:EMM:1850003.1850235,ontology-search} and techniques  such as distributed 
queries\cite{Hartig09executingsparql,Ankolekar07thetwo,sparqlOpt}, scalable reasoning process~\cite{DBLP:conf/semweb/UrbaniKOH09,HoganHarthPolleres2009,DBLP:conf/semweb/HoganPPD10}, 
annotation of web pages~\cite{rdfa-primer} or information retrieval~\cite{Pound} to name a few.

In the particular case of statistical data, the RDF Data Cube Vocabulary~\cite{rdf-data-cube}
a W3C Working Draft document, is a shared effort to represent statistical data in RDF reusing parts (the cube model) 
of the Statistical Data and Metadata Exchange Vocabulary (SDMX)~\cite{sdmx}, an ISO standard 
for exchanging and sharing statistical data and meta-data among organizations. The Data Cube vocabulary is a core 
foundation which supports extension vocabularies to enable publication of other aspects of statistical data flows or 
other multi-dimensional data sets. Previously, the Statistical Core Vocabulary~\cite{scovo} was the standard in fact to describe statistical information in the Web of Data.
Some works are also emerging to mainly publish statistical data following the concepts of the LOD initiative 
such as~\cite{DBLP:conf/semweb/ZapilkoM11,DBLP:journals/ijsc/SalasMBCMA12,DDI2013,DBLP:conf/dgo/FernandezMG11,webindexlod} 
among others.

All the aforementioned works must be considered in order to re-use existing vocabularies and datasets to address 
the challenges of creating meta-described quantitative indexes. Mainly semantics allows us to model logical restrictions 
on data and the computation process while linked data enables the description of indexes in terms of existing concepts and 
the publication of new data and information under a set of principles to boost their re-use and automatic 
processing through machine-readable formats and access protocols.

